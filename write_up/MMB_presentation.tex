\documentclass{beamer}

\usepackage[]{geometry}
% \usepackage{fancyhdr}
% \pagestyle{fancy}
\usepackage[english]{babel}
\usepackage{csquotes,xpatch} % Needs to be loaded after babel
\usepackage[style=authoryear]{biblatex}
\addbibresource{bibliography.bib}

% \usepackage{algorithm}
% \usepackage{algpseudocode}
\usepackage{amsmath,amssymb,amsbsy,amsthm}
\usepackage{graphicx}
% \usepackage{hyperref}
% \usepackage{mathptmx} %Times new romans text
% \usepackage{My_AMA}
\usepackage{pgfplots}
\pgfplotsset{compat=1.18}

% \usepackage{subcaption}
% \usepackage{tikz}
% \usetikzlibrary{arrows.meta,backgrounds}
\graphicspath{ {images} }

\usepackage{xcolor}
\definecolor{ruby}{RGB}{192, 47, 29}

\DeclareMathOperator*{\argmax}{arg\,max}
\DeclareMathOperator*{\argmin}{arg\,min}
\DeclareMathOperator{\cov}{cov}
\DeclareMathOperator{\E}{\mathbb{E}}
\DeclareMathOperator{\Exp}{Exp}
\DeclareMathOperator{\MVN}{MVN}
\newcommand{\N}{\mathcal{N}}
\DeclareMathOperator{\p}{\mathbb{P}}
\DeclareMathOperator{\Pois}{Pois}
\newcommand{\R}{\mathbb{R}}

\makeatletter
\providecommand*{\diff}%
    {\@ifnextchar^{\DIfF}{\DIfF^{}}}
\def\DIfF^#1{%
    \mathop{\mathrm{\mathstrut d}}%
        \nolimits^{#1}\gobblespace}
\def\gobblespace{%
        \futurelet\diffarg\opspace}
\def\opspace{%
        \let\DiffSpace\!%
        \ifx\diffarg(%
            \let\DiffSpace\relax
        \else
            \ifx\diffarg[%
               \let\DiffSpace\relax
            \else
               \ifx\diffarg\{%
                   \let\DiffSpace\relax
               \fi\fi\fi\DiffSpace}


%Information to be included in the title page:
\title[BOLFI]{Bayesian Optimisation for Likelihood Free Inference}
\subtitle{Make model parameterisation go brrr}
\author{Jacob Cumming}
\institute{University of Melbourne}
\date{April 2024}
\logo{\includegraphics[height=1cm]{unimelb_logo}}

\begin{document}

\frame{\titlepage}

\begin{frame}
    \frametitle{Gaussian Processes}
    There you go
\end{frame}


\begin{frame}
    \frametitle{Sample frame title}
    Basic function is $x(x-1)(x+1).$
\end{frame}

\begin{frame}
    \frametitle{Sample frame title}
    \includegraphics[width=\textwidth]{flatish_GP_ell_5_tenths.pdf}
\end{frame}

\begin{frame}
    \frametitle{Sample frame title}
    \begin{figure}
        \centering
        \includegraphics[width=\textwidth]{flatish_GP_ell_10_tenths.pdf}
        \caption{$\sigma = 1$}
    \end{figure}
\end{frame}

\begin{frame}
    \frametitle{Sample frame title}
    \begin{figure}
        \centering
        \includegraphics[width=\textwidth]{flatish_GP_ell_20_tenths.pdf}
        \caption{$\sigma = 1$}
    \end{figure}
\end{frame}


\end{document}