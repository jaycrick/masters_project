\chapter{Conclusion}

Malaria, a significant global health challenge, continues to burden the 
global health system. Mathematical disease models are increasingly being 
harnessed to alleviate this burden. This thesis explores ways to maximise 
the impact of epidemiological models by ensuring they accurately simulate 
public health scenarios, with a specific focus on the complex issue of 
\textit{P. vivax} malaria. 

The complicated lifecycle of \textit{P. vivax} poses a tough hurdle in 
disease modelling,
particularly with respect to asymptomatic cases and relapses.
This complexity is problematic during the parameter 
calibration stage, where traditional methods that rely on being about to 
compute a likelihood are not viable. Likelihood-free methods, while effective, 
come with a high computational cost due to repeated and inefficient model runs. 
This has led some researchers to calibrate models using deterministic 
approximations, failing to capture the stochastic models' uncertainty. 
There is an obvious need for a better calibration methodology.

This research used Gaussian processes to approximate the distribution of the 
discrepancy function used in approximate Bayesian computation for any set of 
parameters. The true likelihood function was approximated by using the 
Gaussian process to create a synthetic likelihood. Empirical evidence 
demonstrated that this methodology successfully recovered parameters of a 
\textit{P. vivax} model given simulated data while mitigating the 
computational overhead. Additionally, this thesis laid out possible further 
extensions to the methodology.

In conclusion, this thesis has demonstrated the plausibility of a new, 
more robust method of calibrating malaria models, particularly for 
\textit{P. vivax.} The use of Gaussian processes and synthetic likelihoods has 
proven effective in overcoming the infeasibility of traditional calibration 
methods and the computational cost of likelihood-free calibration methods. 
Future research should focus on refining these techniques and exploring their 
application to other infectious diseases. These advancements are crucial for 
developing more effective interventions and ultimately to help support 
achieving malaria eradication.