\chapter{Introduction}

Malaria is an infectious disease that poses a significant global health 
challenge. In 2022, the World Health Organisation estimated that malaria 
caused over 600,000 
deaths, most of which were in children under five 
\parencite{world_health_organization_world_2022}. 
The majority of malaria-related deaths are attributable to the 
\textit{Plasmodium falciparum} species of the parasite. 
For this reason, it has been the primary focus of most malaria research. 
However,
with declining \textit{P. falciparum} cases, other malaria species, such as
\textit{P. vivax}, have gained more attention \parencite{price_plasmodium_2020}.
The literature is increasingly recognising that 
the amount of death and severe disease attributable to the 
\textit{Plasmodium vivax} species has 
traditionally been underestimated. 

Various countries with endemic malaria have undertaken significant efforts to 
reduce the burden of or eradicate malaria. Mathematical disease models are 
increasingly aiding these efforts by helping to understand the spread of the 
disease and estimate the effectiveness of possible interventions. 
Malaria models are complex due to the many staged lifecycle, as the parasite 
needs both vertebrate and mosquito hosts. Since malaria is transmitted to humans
through mosquitos, the transmission of malaria within a country is highly 
heterogeneous. Forest workers and those who spend time in areas with high
mosquito density are more likely to contract the disease and may bring
it back to their local community. \textit{P. vivax} has a dormant liver 
stage that is absent in \textit{P. falciparum} and can cause relapses, 
further complicating \textit{P. vivax} models.

To simulate different scenarios using models, modellers must first calibrate
the model parameters to approximate observed disease dynamics. Calibration
is done by 
measuring data such as case counts or current prevalence surveys. 
Standard techniques
to calibrate parameters include maximum likelihood estimation and sampling
from a posterior distribution, which both require a likelihood function. As
models become increasingly complicated, analytic forms for the likelihood may
not exist, or calculating the likelihood may be very computationally
burdensome. Some researchers calibrate compartmental models by relying on
their deterministic counterparts, which is questionable, as the deterministic
model sometimes behaves differently from the stochastic model.

Modern
likelihood-free techniques, such as approximate Bayesian computation, 
have been designed to
facilitate a principled parameter calibration method.
Various forms of approximate Bayesian computation have been widely adopted. 
However, all forms 
require large numbers of model runs, which may be unfeasible or undesirable.

Non-parametric regression techniques such as Gaussian process regression
have been developed for a long time and used in statistical 
modelling for decades (see \cite{diggle_analysis_1994}), However,
their uses in disease modelling
have yet to be fully explored. Gaussian processes place a prior 
distribution over a function space. Observations from the true function can
create a posterior distribution over the same function space, 
inherently accounting for uncertainty for unobserved values.

Drawing on concepts from approximate Bayesian computation, we aim to improve
parameter calibration in malaria models, addressing a recognised need for
improvement. We do this by training a Gaussian process to predict how close a
model run will be to observed data. The Gaussian process approximation
can then be used to extract a synthetic likelihood which approximates the
true unknown likelihood. This allows for the use of frequentist and Bayesian
parameter inference.

The thesis follows the following outline: The literature review in Part I
discusses fundamental concepts of epidemiological modelling, covering
deterministic ordinary differential equation models, stochastic models, and
their simulation methods. It then examines and reviews various malaria
models to assess how \textit{P. vivax} is modelled in the
current literature. 
Part I further surveys parameter inference techniques for disease and other
models, comparing
frequentist and Bayesian approaches, and ends with discussing likelihood-free
techniques, particularly approximate Bayesian computation.
Approximate Bayesian computation then motivates the use of Gaussian processes 
regression, which enables the development of a synthetic likelihood. 
The synthetic likelihood can be used instead of the unknown true
likelihood to use the traditional parameter calibration techniques in complex
models. Finally, Part I discusses using Bayesian acquisition functions to 
efficiently train
the Gaussian process regression.

Part II applies and extends this methodology by calibrating parameters specific
to a \textit{P. vivax} model using synthetic observed data. The results and 
discussion
section validates the method and demonstrates that the parameters that produced
the observed data are recoverable. Finally, the thesis concludes with a
discussion of the findings and outlines avenues for future research.