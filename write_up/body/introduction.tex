\chapter{Introduction}

Malaria is an infectious disease that poses a significant global health 
challenge. In 2022, the WHO estimated that malaria had caused over 600,000 
deaths that year, most of which were in children under five. 
The majority of malaria-related deaths are attributable to the 
\textit{Plasmodium falciparum} species of the parasite. 
However, the literature is increasingly recognising that 
the amount of death and severe disease attributable to the 
\textit{Plasmodium vivax} species is 
likely traditionally underestimated. The malaria lifecycle is complicated, 
needing both a vertebrate and mosquito host. In addition, \textit{P. vivax} has 
a dormant stage that causes relapses. 

Various countries with endemic malaria have undertaken significant efforts to 
reduce the burden of or eradicate malaria. Mathematical disease models are 
increasingly aiding these efforts by helping to understand the spread of the 
disease and estimate the effectiveness of possible interventions. 
Malaria models are complex due to the many staged lifecycle, and in 
particular, \textit{P. vivax} models need to consider relapses, further 
complicating the model.

To simulate different scenarios using models, modellers must first calibrate
the model parameters to approximate observed disease dynamics, measured by
observed data such as case counts or prevalence surveys. Standard techniques
to calibrate parameters include maximum likelihood estimation and sampling
from a posterior distribution, which both require a likelihood function. As
models become increasingly complicated, analytic forms for the likelihood may
not exist, or calculating the likelihood may be very computationally
burdensome. Some researchers calibrate compartmental models by relying on
their deterministic counterparts, which is questionable, as the deterministic
model sometimes behaves differently from the stochastic model. Modern
likelihood-free techniques, such as approximate Bayesian computation, address
this issue but require large model runs to calibrate the parameters.

Drawing on concepts from approximate Bayesian computation, we aimed to improve
parameter calibration in malaria models, addressing a recognised need for
improvement. We do this by training a Gaussian process to predict how close a
model run will be to observed data and extracting a synthetic likelihood.

The thesis follows the following outline: The literature review in Part I
discusses fundamental concepts of epidemiological modelling, covering
deterministic ordinary differential equation models, stochastic models, and
their simulation methods. It then examines malaria and reviews various malaria
models. Part I further surveys parameter inference techniques, comparing
frequentist and Bayesian approaches, and ends with discussing likelihood-free
techniques, particularly approximate Bayesian computation.
Approximate Bayesian computation then motivates using Gaussian processes to
develop a synthetic likelihood, which can be used in place of the unknown true
likelihood to use the traditional parameter calibration techniques in complex
models. Finally, Part I discusses using Bayesian acquisition functions to train
the Gaussian process efficiently.

Part II applies and extends this methodology by calibrating parameters specific
to a \textit{P. vivax} model using synthetic observed data. The results and 
discussion
section validates the method and demonstrates that the parameters that produced
the observed data are recoverable. Finally, the thesis concludes with a
discussion of the findings and outlines avenues for future research.